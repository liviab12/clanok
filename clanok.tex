% Metódy inžinierskej práce

\documentclass[10pt,twoside,slovak,a4paper]{article}

\usepackage[slovak]{babel}
%\usepackage[T1]{fontenc}
\usepackage[IL2]{fontenc} % lepšia sadzba písmena Ľ než v T1
\usepackage[utf8]{inputenc}
\usepackage{graphicx}
\usepackage{url} % príkaz \url na formátovanie URL
\usepackage{hyperref} % odkazy v texte budú aktívne (pri niektorých triedach dokumentov spôsobuje posun textu)

\usepackage{cite}
%\usepackage{times}

\pagestyle{headings}

\title{Algoritmy odporúčania príspevkov v aplikácii Instagram\thanks{Semestrálny projekt v predmete Metódy inžinierskej práce, ak. rok 2024/25, vedenie: Mgr. Martin Sabo, PhD.}} % meno a priezvisko vyučujúceho na cvičeniach

\author{Lívia Bajzová\\[2pt]
	{\small Slovenská technická univerzita v Bratislave}\\
	{\small Fakulta informatiky a informačných technológií}\\
	{\small \texttt{xbajzova@stuba.sk}}
	}

\date{\small 30. september 2024} % upravte



\begin{document}

\maketitle

\begin{abstract}
Cieľom článku je preskúmať a analyzovať odporúčacie systémy, ktoré Instagram používa na personalizáciu obsahu v jeho hlavných častiach: Feed, Explore a Reels. Článok sa zameria na to, ako algoritmy spracovávajú používateľské dáta, interakcie a preferencie na základe predchádzajúceho správania používateľov, aby efektívne odporúčali relevantný obsah.

Prvá časť článku sa zameria na odporúčacie algoritmy sociálnych sietí, kde vysvetlíme ako všeobecne funguje odporúčanie obsahu na sociálnych sieťach, ktoré sa prejavuje aj v aplikácii Instagram. Na vysvetlenie tohto procesu predstavíme modely šírenia informácií a vysvetlíme metódy strojového učenia.

V druhej časti článku sa budeme venovať časti aplikácie zvanej Feed, kde sa používateľom zobrazujú príspevky účtov, ktoré sledujú, doplnené príspevkami iných účtov, o ktoré môžu mať používatelia Instagramu záujem, a sponzorovanými príspevkami.

Tretia časť sa zameria na Explore, kde preskúmame, aké techniky a algoritmy Instagram používa na identifikáciu nového obsahu, ktorý by mohol používateľov zaujať.

Nakoniec sa zameriame na Reels, kde budeme analyzovať, aké faktory prispievajú k populárnosti a odporúčaniu videí. Preskúmame, ako algoritmy zohľadňujú trendy, dĺžku videí a interakcie používateľov pri hodnotení relevancie obsahu.



\end{abstract}



\section{Úvod}
Sociálne siete sa stali jednou z najdominantnejších platforiem, ktoré denne ovplyvňujú milióny ľudí po celom svete. Jednou z hlavných výziev, ktorým čelia takéto platformy, je udržanie pozornosti používateľov a poskytovanie obsahu, ktorý je pre nich osobne relevantný. Instagram, ako jedna z najpopulárnejších sociálnych sietí, využíva na tento účel odporúčacie systémy. Tieto algoritmy analyzujú správanie používateľov a ich interakcie s obsahom, aby mohli personalizovať obsah a zlepšovať používateľský zážitok. Základné fungovanie odporúčacích systémov sociálnych sietí, ktoré sa týkajú aj samotného Instagramu je vysvetlené v časti~\ref{druha}.

Instagram nemá jeden jedinečný algoritmus pre celú aplikáciu, každá časť aplikácie - Feed, Explore a Reels používa svoj vlastný algoritmus pre odporúčanie obsahu, prispôsobený tomu ako ju ľudia používajú a čo udrží ich pozornosť v tejto sekcii aplikácie najdlhšie. Podrobnejšie súvislosti týkajúce sa algoritmov v daných častiach Instagramu sú uvedené postupne v častiach~\ref{tretia},~\ref{stvrta} a~\ref{piata}. Záverečné zhrnutie najdôležitejších súvislostí a zistení prináša časť~\ref{siesta}.

Prostredníctvom tejto analýzy jednotlivých častí aplikácie chceme poskytnúť ucelený prehľad o tom, ako Instagram formuje používateľskú skúsenosť a aké techniky, nástroje a prístupy využíva na dosiahnutie efektívneho a personalizovaného odporúčania obsahu.




\section{Odporúčacie algoritmy sociálnych sietí} \label{druha}

\subsection{Typy šírenia informácií na sociálnych sieťach} 
Sociálne siete medzi ktoré patrí aj aplikácia Instagram využívajú na šírenie obsahu na platforme tri typy šírenia informácií - odber, sieť a algoritmus (1).

Model odberu predstavuje situáciu, kedy sa príspevok dostane k tým používateľom, ktorý odberajú daný profil na sociálnej sieti. Na Instagrame sa nám teda zobrazí príspevok profilu, ktorý máme medzi sledovanými účtami.

V sieťovom modeli dochádza k šíreniu obsahu sieťou prostredníctvom zdieľania toho istého príspevku ďalším používateľom na svojom profile. V aplikácii Instagram to predstavuje situáciu kedy označíme v nami pridanom príspevku aj iných používateľov a následne sa náš príspevok dostáva aj k ich sledovateľom.

Algoritmický model odporúča príspevky o ktorých algoritmus predpovedá, že s najväčšou pravdepodobnosťou zaujmú používateľa a motivujú ho k ďalšej aktivite na platforme. Využíva na to metódy strojového učenia a v súčasnosti patrí medzi popredný model rozhodujúci o spôsobe odporúčania obsahu na sociálnych sieťach.

\begin{figure*}[tbh]
\centering
\includegraphics[scale=.5]{3models.png}
\caption{Modely šírenia informácií na sociálnych sieťach: odber, sieť, internet (1)}
\end{figure*}

\subsection{Metódy strojového učenia} 
Odporúčacie algoritmy zohrávajú zásadnú úlohu v tom, ako sa používateľom sociálnych sietí filtruje a objavuje relevantný obsah na platforme. Tieto algoritmy pracujú na základe údajov o správaní používateľov ako sú interakcie, lajky, komentáre, prezeranie profilov, zdieľanie príspevkov, či čas strávený sledovaním určitých videí a vytvárajú sa tak personalizované odporúčania pre každého konkrétneho používateľa s cieľom udržať ho na platforme čo najdlhšie. 

V pozadí tohto procesu stoja metódy strojového učenia:
\begin{itemize}
\item kolaboratívne filtrovanie
\item obsahové filtrovanie
\item hybridné filtrovanie
\item demografické filtrovanie (2)
\end{itemize}



\begin{figure*}[tbh]
\centering
\includegraphics[scale=.5]{diagram.png}
\caption{Odporúčanie obsahu na sociálnych sieťach}
\end{figure*}

Kolaboratívne filtrovanie je metóda odporúčania, ktorá analyzuje správanie podobných používateľov s cieľom odporučiť obsah inému používateľovi na základe toho, čo sa páči ostatným používateľom s podobnými záujmami (3). Napríklad, ak sa veľa používateľov, ktorí lajkujú a sledujú podobné príspevky ako daný používateľ, začne zaujímať o konkrétny typ obsahu (napr. príspevky s určitým hashtagom), algoritmus to zachytí a ponúkne tento obsah aj jemu. Týmto spôsobom sa Instagram snaží predpovedať, čo by mohlo používateľa zaujať.

Obsahové filtrovanie sa zameriava na vlastnosti obsahu, s ktorým používateľ už interagoval, pričom algoritmus analyzuje a vyberá ďalší obsah so zhodnými charakteristikami. To v praxi znamená, že vyberá ďalšie príspevky na základe ich podobnosti s inými príspevkami, ktoré používateľa zaujali v minulosti. (4) Ak napríklad používateľ často lajkuje príspevky s cestovateľskými fotografiami alebo sleduje videá o varení, algoritmus Instagramu zaznamená tieto preferencie a následne zobrazí viac podobného obsahu. Na kategorizáciu a identifikáciu príspevkov a videí s podobným obsahom využíva hashtagy ako aj samotný popis príspevku. Výsledkom je, že používateľ vidí častejšie príspevky totožné s jeho záujmami a preferenciami.

Hybridné filtrovanie kombinuje kolaboratívne a obsahové filtrovanie za účelom väčšej presnosti a lepšej účinnosti odporúčacieho procesu. Súčasne analyzuje podobnosť správania medzi používateľmi a podobnosť medzi zdieľaným obsahom a nami sledovaným obsahom a následne hľadá najväčšiu zhodu a tým predpovedá čo môže používateľa najviac zaujať (5).

Demografické filtrovanie využíva na odporúčanie obsahu demografické údaje poskytnuté používateľom, ako je vek, pohlavie, poloha či jazyk. Hlavná nevýhoda tohoto typu strojového učenia spočíva v tom, že ľudia v online priestore nechcú zdieľať svoje osobné údaje alebo ich neuvádzajú pravdivo zo strachu z ich zneužitia (6). V praxi to znamená, že algoritmus môže pri odporúčaní príspevkov zohľadniť konkrétne vlastnosti používateľa. Napríklad používateľom v určitej vekovej skupine sa môžu častejšie zobrazovať trendy príspevky alebo videá populárne medzi ich vekovou skupinou, zatiaľ čo poloha môže ovplyvniť zobrazovanie miestnych udalostí, podnikov alebo obsahu v ich jazyku. Demografické filtrovanie tak umožňuje personalizovať odporúčaný obsah každého používateľa v kontexte jeho širšieho demografického profilu.


\section{Feed} \label{tretia}




\section{Explore} \label{stvrta}




\section{Reels} \label{piata}



\section{Záver} \label{siesta} % prípadne iný variant názvu
\section{Literatúra} 


%\acknowledgement{Ak niekomu chcete poďakovať\ldots}


% týmto sa generuje zoznam literatúry z obsahu súboru literatura.bib podľa toho, na čo sa v článku odkazujete
\bibliography{literatura}
\bibliographystyle{plain} % prípadne alpha, abbrv alebo hociktorý iný
\end{document}
