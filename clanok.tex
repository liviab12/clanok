% Metódy inžinierskej práce

\documentclass[10pt,twoside,slovak,a4paper]{article}

\usepackage[slovak]{babel}
%\usepackage[T1]{fontenc}
\usepackage[IL2]{fontenc} % lepšia sadzba písmena Ľ než v T1
\usepackage[utf8]{inputenc}
\usepackage{graphicx}
\usepackage{url} % príkaz \url na formátovanie URL
\usepackage{hyperref} % odkazy v texte budú aktívne (pri niektorých triedach dokumentov spôsobuje posun textu)

\usepackage{cite}
%\usepackage{times}

\pagestyle{headings}

\title{Algoritmy odporúčania príspevkov v aplikácii Instagram\thanks{Semestrálny projekt v predmete Metódy inžinierskej práce, ak. rok 2024/25, vedenie: Mgr. Martin Sabo, PhD.}} % meno a priezvisko vyučujúceho na cvičeniach

\author{Lívia Bajzová\\[2pt]
	{\small Slovenská technická univerzita v Bratislave}\\
	{\small Fakulta informatiky a informačných technológií}\\
	{\small \texttt{xbajzova@stuba.sk}}
	}

\date{\small 30. september 2024} % upravte



\begin{document}

\maketitle

\begin{abstract}
Cieľom článku je preskúmať a analyzovať odporúčacie systémy, ktoré Instagram používa na personalizáciu obsahu v jeho hlavných častiach: Feed, Explore a Reels. Článok sa zameria na to, ako algoritmy spracovávajú používateľské dáta, interakcie a preferencie na základe predchádzajúceho správania používateľov, aby efektívne odporúčali relevantný obsah.

V prvej časti článku sa budeme venovať časti aplikácie zvanej Feed, kde sa používateľom zobrazujú príspevky účtov, ktoré sledujú, doplnené príspevkami iných účtov, o ktoré môžu mať používatelia Instagramu záujem, a sponzorovanými príspevkami. Preskúmame, ako Instagram analyzuje interakcie s príspevkami, aby vytvoril personalizovaný prúd obsahu, a aké faktory ovplyvňujú jeho algoritmus, vrátane času stráveného na príspevku a počtu lajkov a komentárov.

Druhá časť sa zameria na Explore, kde preskúmame, aké techniky a algoritmy Instagram používa na identifikáciu nového obsahu, ktorý by mohol používateľov zaujať. Bude sa skúmať, ako sa algoritmy učia z užívateľských preferencií a akú úlohu zohráva strojové učenie pri odporúčaní obsahu.

Nakoniec sa zameriame na Reels, kde budeme analyzovať, aké faktory prispievajú k populárnosti a odporúčaniu videí. Preskúmame, ako algoritmy zohľadňujú trendy, dĺžku videí a interakcie používateľov pri hodnotení relevancie obsahu.

\end{abstract}



\section{Úvod}



\section{Odporúčacie algoritmy sociálnych sietí} \label{nejaka}


\section{Feed} \label{ina}




\section{Explore} \label{dolezita}




\section{Reels} \label{dolezitejsia}



\section{Záver} \label{zaver} % prípadne iný variant názvu
\section{Literatúra} 


%\acknowledgement{Ak niekomu chcete poďakovať\ldots}


% týmto sa generuje zoznam literatúry z obsahu súboru literatura.bib podľa toho, na čo sa v článku odkazujete
\bibliography{literatura}
\bibliographystyle{plain} % prípadne alpha, abbrv alebo hociktorý iný
\end{document}
